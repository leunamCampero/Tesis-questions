\documentclass[letterpaper,10pt,english,spanish]{article}
%\documentclass[spanish]{article}

\usepackage[latin1]{inputenc} % Caracteres acentuados y la �. 
\usepackage[dvips]{epsfig}    % Incluir figuras en PostScript
\usepackage{color} 
\usepackage{babel}            % Distintas lenguas (en este caso spanish)
\usepackage{pslatex}          % fonts de postscript, para generar PDFs
\usepackage{url}              % Colocar una url
\usepackage{amssymb}          % Notacion numeros complejos,Reales...
\usepackage{charter}  
\usepackage{pgfkeys}        % Tipo de letra
%\usepackage{algorithm}       % paquete para usar algorithm
%\usepackage{algorithmic}     % paquete para usar algorithm
%\usepackage{fancyheadings}    % Usar comentario en pie y cabeza
%\usepackage{fullpage}        % Llena mas las paginas
%\usepackage[mathcal]{euscript}

%%%%% Para hacer referencias (mas abajo hay un ejemplo, esta comentado) 
% \newcommand{\water}{$H_2O$}  % Definir un nuevo comando \water


%%%%%%%%%  Para que diga ``Tabla 1'' en vez de ``Cuadro 1''
%\addto\captionsspanish{  
%\def\listtablename{Indice de Tablas}
%\def\tablename{Tabla}
%}
    

%%%%%% Encabezado de pie y cabeza (necesita paquete fancyheadings)
%%\pagestyle{fancyplain}
%%\lhead[\fancyplain{}{\bf P�g. \thepage}]{\fancyplain{}{\bf Plantilla
%%    de Ejemplo}}  
%%\rhead[\fancyplain{}{\bf Plantilla de ejemplo}]{\fancyplain{}{\bf 
%%    P�gina \thepage }} 
%%\lfoot[\fancyplain{}{\bf \today}]{} 
%%\cfoot{}
%%\rfoot[]{\fancyplain{}{\bf \today}}
%%\footrulewidth 0.4pt

%%\makeatother
%%%%%%%%%

\title{Preguntas relacionadas con la tesis t�tulada "Un enfoque computacional a la representaci�n 
del grupo sim�trico en homolog�as"}
\author{Manuel Campero Jurado en colaboraci�n con el Dr. Rafel Villarroel Flores}

\date{\today} % si no la pongo se pone automaticamente
  
\begin{document}
\maketitle
\thispagestyle{empty} % Primera pagina sin numero
  
 

\section{Preguntas}
En el art�culo Equivariant collapses and the homotopy type of iterated clique graphs en la segunda p�gina se dice que si $G$ es una gr�fica y $x$ es un v�rtice, entonces se denota por $N_G(x)$ al conjunto de todos los vecinos de $x$ en $G$. La vecindad cerrada de $x$ es $N_G\left[ x \right] = N_G(x) \cup \left\{ x \right\}$ y si $G$ es una gr�fica y $X  \subseteq G$, se denota la vecindad com�n cerrada de $X$ en $G$ como $N_G\left[ X \right] = \cap_{x \in X} N_G\left[ x \right]$ y posteriormente en la proposici�n 2.2 se utiliza $N_G\left[  \sigma \right]$ donde $\sigma$ es una cara de un subcomplejo, as� que mi pregunta es �C�mo se define $N_G\left[  \sigma \right]$ (ya que una cara no necesariamente es un subconjunto del 1-esqueleto de un complejo simplicial)? y posteriormente se habla sobre $N_G\left[ e \right]$ donde $e$ es una arista �$N_G\left[ e \right]$ denota a los al conjunto cerrado de vecinos de los v�rtices de $G$ que est�n conectados por $e$? o �c�mo ser�a la definici�n general del conjunto cerrado de vecinos de $e$ en ese caso?								

Posteriormente se dice que si $\mathcal{F}$ es una familia de caras libres de un complejo simplicial $\Delta$ diremos que es independientemente libre si simpre que $\sigma$, ${\sigma}' \in {F}$ (por cierto �la contenci�n es propia?) y $\tau \in \Delta$ son tal que $\sigma \subseteq \tau$ y ${\sigma}' \subseteq \tau$, entonces ${\sigma}' = \sigma$.
Y se denota $[ \mathcal{F},\infty )  = \bigcup_{\sigma \in \mathcal{F}}  [ \sigma,\infty ) $ %Disuculpe no encontr� como hacer intervalos semi-abiertos o semi-cerrados en LaTex

Luego (4) de la proposici�n 2.4 dice que si $\mathcal{F}$ es finito y independientemente libre, entonces $\Delta$ se colapsa en $\Delta \setminus [ \mathcal{F},\infty ) $.

Luego vienen dos condiciones para decir que un grupo $\Gamma$ act�a simplicialmente en un complejo simplicial $\Delta$:
\begin{enumerate}
\item Si $x, gx \in \varphi$ para alg�n $g \in \Gamma$, $\varphi \in \Delta$, entonces $x = gx$.
\item Si $g_0, g_1, \ldots, g_n \in \Gamma$ y $\left \{ x_0, x_1, \ldots, x_n \right \}$, $\left \{ g_0 x_0, g_1 x_1, \ldots, g_n x_n \right \} \in \Delta$, entonces existe un $g \in \Gamma$ tal que $g x_i = g_i x_i$ para todo $i$.
\end{enumerate}

La proposici�n 2.5 dice que sea $\Delta$ un $\Gamma -$ complejo que satisface $1$. Sean $\sigma, \varphi \in \Delta$ tales que $\sigma, g \sigma \subset \varphi$ para alg�n $g \in \Gamma$. Entonces $\sigma = g \sigma$. En particular, siempre que $\sigma$ es una cara libre de $\Delta$, la $\Gamma -$ �rbita de $\sigma$, $\Gamma \sigma = \left \{ g \sigma \mid  g \in \Gamma \right \}$ es independientemente libre.

Adem�s dado $\Delta$ un $\Gamma -$ complejo. Para un v�rtice $x$ de $\Delta$ sea $\Gamma x = \left \{ g x \mid  g \in \Gamma \right \}$. La �rbita compleja $\Delta / \Gamma$ tiene como conjunto de v�rtices $\Gamma x = \left \{ g x \mid  g \in V(\Delta) \right \}$ y simplejos $\Delta / \Gamma = \left \{ p(\sigma) \mid \sigma \in \Delta \right \}$ donde $p: V(\Delta) \rightarrow  V(\Delta / \Gamma)$ est� dado por $x \mapsto\Gamma x$.

Y la proposici�n 2.6 dice que dada $\sigma$ una cara libre de alg�n $\Gamma -$ complejo $\Delta$ con la propiedad 2, entonces $p(\sigma)$ es una cara libre  de $\Delta / \Gamma$ y $(\Delta \setminus [ \Gamma \sigma,\infty ) / \Gamma = (\Delta / \Gamma) [p(\sigma),\infty)$.

En el segundo parr�fo de la p�gina 4 dice que si $\sigma$ es una cara libre del $\Gamma -$ complejo $\Delta$ y su $\Gamma -$�rbita $\left \{ g \sigma \mid  g  \Gamma \right \}$ es independientemente libre, se dir� que $(\Delta \setminus [ \Gamma \sigma,\infty )$ se obtiene de $\Delta$ por un elemental $\Gamma -$ colapso. Con lo anterior, no s� si estoy entendiendo bien, en mi caso �estoy buscando $(\Delta \setminus [ \Gamma \sigma,\infty )$?, es decir �$(\Delta \setminus [ \Gamma \sigma,\infty )$ tiene la misma informaci�n (por as� decirlo) que $\Delta$ al actuar simplicialmente $\Gamma$ sobre ellos?.
							

\end{document}

