\documentclass[letterpaper,10pt,english,spanish]{article}

\usepackage[latin1]{inputenc} % Caracteres acentuados y la �. 
\usepackage[dvips]{epsfig}    % Incluir figuras en PostScript
\usepackage{color} 
\usepackage{babel}            % Distintas lenguas (en este caso spanish)
\usepackage{pslatex}          % fonts de postscript, para generar PDFs
\usepackage{url}              % Colocar una url
\usepackage{amssymb}          % Notacion numeros complejos,Reales...
\usepackage{charter}          % Tipo de letra


\title{Preguntas relacionadas con la tesis t�tulada ``Un enfoque computacional a la representaci�n 
del grupo sim�trico en homolog�as''}
\author{Manuel Campero Jurado en colaboraci�n con el Dr. Rafel Villarroel Flores}

\date{\today} % si no la pongo se pone automaticamente
  
\begin{document}
\maketitle
\thispagestyle{empty} % Primera pagina sin numero
  
 

\section{Preguntas}

Buenas tardes Doctor, he estado estudiando los art�culos que hizo el favor de mandarme, sigo estudiando ``Equivariant collapses and the homotopy type of iterated clique graphs'' porque tengo algunas dudas sobre los colapsos, pero aun no lo mando porque quiero hacer bien las preguntas, por el momento s�lo quisiera comentarle que para formar los simplejos de un complejo simplicial abstracto yo busco todos los subconjuntos de una ``facet'' (los simplejos que no son caras de simplejos m�s grandes), pero en caso de la homolog�a de $K(M_9)$ ya no puedo calcular los subconjuntos de una sola facet (y $K(M_9)$ tiene m�s de 55,000 facet), tambi�n intent� usar la funci�n de NetworksX ``enumerate all cliques(G)'' (claro que en el caso particular cuando el complejo simplicial proviene de un gr�fica), pero tampoco termina, �usted conoce o tiene una alguna funci�n que si permita calcular todos lo simplejos del complejo simplicial abstracto asociado con $K(M_9)$?, ya que lo que llevo del art�culo supone que yo conozco todos los simplejos de un complejo simplicial (porque necesito saber cuales caras son libres, etc).

En el art�culo que hizo el favor de mandarme (The clique operator on matching and chessboard graphs) el corolario 4.3 dice que si una gr�fica $G$ tiene a lo m�s $8$ v�rtices, entonces su gr�fica de l�neas $L(G)$ satisface que $K(L(G)) \simeq  L(G)$. Sus tesistas anteriores y yo hemos buscado para qu� $n,k \geq 0$ se cumple $\tilde{H_k}(M_n) \cong_{s_n} \tilde{H_k}(K(M_n))$ (aunque en mi caso yo encuentro la homolog�a de dimensi�n cero y no la homolog�a reducida de dimensi�n cero), por eso tengo otra pregunta: �es posible que $\tilde{H_k}(M_n) \cong_{s_n} \tilde{H_k}(K(M_n))$ para $n> 9$ y todo $k \geq 0$?* o �eso contradice el corolario anterior?

Lo siguiente es s�lo si la respuesta a * es negativa:
Program� en python el teorema de Bouc y me dice que 
$\tilde{H_k}(M_9) \cong_{s_n} 0$ para $k \in \left \{ 0, 1, 4, 5, \ldots  \right \}$,  $\tilde{H_2}(M_9) \cong_{s_n} S^{(3,3,3)}$ y $\tilde{H_3}(M_9) \cong_{s_n} S^{(5,1,1,1,1)}$.

Entonces como la respuesta a * no es afirmativa, en principio $\tilde{H_2}(K(M_9))$ y $\tilde{H_3}(K(M_9))$ como $S_n$ m�dulos podr�an ser isomorfas a muchas cosas �cierto?, pero �se debe cumplir que $\tilde{H_2}(K(M_9)) \ncong_{s_n} S^{(3,3,3)}$ o $\tilde{H_3}(K(M_9)) \ncong_{s_n} S^{(5,1,1,1,1)}$? (pensando que si para alg�n $k \geq 0$ se tiene que $\tilde{H_k}(M_9) \cong_{s_n} 0$, entonces necesariamente $\tilde{H_k}(K(M_9)) \cong_{s_n} 0$), y tal vez est� incluido en lo anterior, pero podr�a pasar que � $\tilde{H_2}(K(M_9)) \cong_{s_n} 0$ o $\tilde{H_3}(K(M_9)) \cong_{s_n} 0$ aun cuando la 2 homolog�a y la 3 homolog�a para $M_9$ es distinta de cero? 



\end{document}

